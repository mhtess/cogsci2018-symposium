% 
% Annual Cognitive Science Conference
% Sample LaTeX Two-Page Summary -- Proceedings Format
% 

% Original : Ashwin Ram (ashwin@cc.gatech.edu)       04/01/1994
% Modified : Johanna Moore (jmoore@cs.pitt.edu)      03/17/1995
% Modified : David Noelle (noelle@ucsd.edu)          03/15/1996
% Modified : Pat Langley (langley@cs.stanford.edu)   01/26/1997
% Latex2e corrections by Ramin Charles Nakisa        01/28/1997 
% Modified : Tina Eliassi-Rad (eliassi@cs.wisc.edu)  01/31/1998
% Modified : Trisha Yannuzzi (trisha@ircs.upenn.edu) 12/28/1999 (in process)
% Modified : Mary Ellen Foster (M.E.Foster@ed.ac.uk) 12/11/2000
% Modified : Ken Forbus                              01/23/2004
% Modified : Eli M. Silk (esilk@pitt.edu)            05/24/2005
% Modified : Niels Taatgen (taatgen@cmu.edu)         10/24/2006
% Modified : David Noelle (dnoelle@ucmerced.edu)     11/19/2014

%% Change "letterpaper" in the following line to "a4paper" if you must.

\documentclass[10pt,letterpaper]{article}

\usepackage{cogsci}
\usepackage{pslatex}
\usepackage{apacite}


\title{Generalizations, from representation to transmission}
 
\author{{\large \bf Michael Henry Tessler (mhtessler@stanford.edu)}, {\large \bf Noah D. Goodman (ngoodman@stanford.edu)}  \\
  Department of Psychology, Stanford University \vspace{-0.1cm}
   \AND {\large \bf David Danks (ddanks@cmu.edu)} \\
  Departments of Philosophy and Psychology, Carnegie Mellon University \vspace{-0.1cm}
    \AND{\large \bf Emily Foster-Hanson (emily.fosterhanson@nyu.edu)}, {\large \bf Marjorie Rhodes (marjorie.rhodes@nyu.edu)} \\
  Department of Psychology, New York University \vspace{-0.1cm}
    \AND {\large \bf Gregory Carlson (calrson@ling.rochester.edu)} \\
  Department of Linguistics, University of Rochester \vspace{-0.1cm}
  }


\begin{document}

\maketitle

%\begin{quote}
%\small
%\textbf{Keywords:} 
%add your choice of indexing terms or keywords; kindly use a
%semicolon; between each term
%\end{quote}


\begin{quote}
\footnotesize
\emph{Because any object or situation experienced by an individual is unlikely to recur in exactly the same form and context, psychology's first general law should, I suggest, be a law of generalization.}  -- Roger Shepard, 1987
\end{quote}
%\vspace{-0.5cm}
\section{Overview}

Generalizable knowledge is crucial to thrive in our open-ended, dynamic world. 
We organize this knowledge with concepts: Understanding that an entity is a \textsc{dog} affords inferences about its properties and behaviors. 
But not all generalizations are straightforward to acquire through direct experience: Relevant observations may be costly (e.g., learning that a plant is poisonous) or rare (e.g., understanding that lightning strikes tall objects). 
It thus is important that language allow us to communicate such generalizations to each other. 
%By sharing generalizations, we flourish collectively without individually needing to taste potentially-poisonous plants or personally witness many lightning strikes.
%The dark side of acquiring generalizations from others is that evidence can be exaggerated: Knowledge about particular categories (e.g., social categories) may be distorted for ulterior motives. 

Thirty years after Shepard's insightful proclamation, the cognitive science of generalization is an expansive topic, covering representational questions concerning the format and organization of concepts to how generalizable knowledge is transmitted from one intelligent creature to the next.
% and how socially-transmitted information shapes and misshapes our theories of the world.
This set of issues cuts across interests of many different areas of cognitive science: from linguists concerned with the truth-conditions of utterances conveying generalization to developmental psychologists studying how children learn generalizable knowledge to computer scientists attempting to build machines that think and talk like people. 
%Because of this breadth and of recent developments in both empirical and theoretical paradigms, we are confident this symposium will be of broad interest at Cognitive Science.
The aim of this symposium is to gather and integrate several distinct empirical and theoretical perspectives on the study of generalization, with a particular focus on the transmission using language.

The symposium will consist of four talks by experts representing philosophy, linguistics, computational modeling, and developmental psychology.    
Danks will argue for a particular representation of generalizations (operations on graphical models) and describe how language interfaces with this representation.
 Tessler presents a probabilistic, cognitive model of understanding language that conveys generalization: \emph{generic language} or generics.
 Foster-Hanson demonstrates in young children how generic language implicit conveys category information even when it appears to be conveying no information.
% semantically negative information, and how even refutations of generalizations can reinforce latent beliefs about categories in young children.
 Carlson investigates the nature of \emph{exceptions} to generic statements (i.e., entities that do not conform to the generalization), and their implications for theories of generics.
We will close with a panel discussion on some of these issues, led by Noah Goodman and Marjorie Rhodes.

\noindent\textbf{Generalizations from the integration of concepts \& (linguistic) structure} 
%\subsubsection{Generalizations from the integration of concepts \& (linguistic) structure}
\noindent\emph{David Danks}

Concepts provide one key mechanism by which people generalize: understanding an object as a \textsc{dog}, for example, enables a number of inferences about the object on the basis of it belonging to this type. 
Similarly, use of a single word by a speaker can prompt generalizations and inferences by a hearer, particularly when the word is (appropriately) connected with the hearer's concepts. 
However, just as our cognitive representations are richer than isolated concepts, language enables the transmission (in some sense) of much more complex information than can be conveyed with single words or terms. 
The generalizations and inferences supported by a full sentence extend far beyond the generalizations supported by each term in the sentence. 

In this talk, I will first argue that many of our concepts can be fruitfully understood as graphical models, and the associated generalizations and inferences understood as operations on those representations. 
I will then show how to integrate these representations (as graphical models) into a unified model of linguistic content that incorporates both ``conceptual content'' and ``structural content'', as well as complex interactions between the two. 
Throughout, I will focus on the linguistic phenomenon of \emph{bridging}, as the generalizations and inferences in those instances depend crucially on both types of content, and so require an integrated model.

\noindent\textbf{The language of generalization, in computational terms} \\
\noindent\emph{Michael Henry Tessler}

%\noindent\emph{Michael Henry Tessler}
% Probabilistic cognitive models formalize generalizations using the Bayesian probability calculus and have been immensely successful at characterizing human thought in computational terms. 

Generalizations are central to human understanding and language provides simple ways to convey them in the form of \emph{generic language}, or generics (e.g., ``Birds fly''). 
Generics are ubiquitous in everyday discourse and child-directed speech, yet the meaning of these expressions is logically puzzling (e.g., not all birds fly) and has resisted precise formalization. 
The major issue in formalizing generic language is determining which statements are true, which are false, and how to interpret them in context. 

Using a probabilistic model of language understanding, I explore the hypothesis that the meaning of these linguistic expressions is simple but underspecified, and that general communicative principles can be used to establish a more precise meaning in context. 
This theory predicts that background knowledge---in the form of prior beliefs about the prevalence of the feature in different categories---is the connective tissue between the flexible endorsements and interpretations of generic sentences. 
To test this theory, we examine generics about novel causal domains while manipulating background knowledge about the causal systems. 
We find that both interpretations and endorsements are sensitive to background knowledge in the ways predicted by the probabilistic model. 
These results suggest that the context-sensitivity of generalizations in language emerges from the interaction of an underspecified meaning with diverse beliefs about properties.

%Generalizations are the foundation of abstract thought: They allow us to make sense of the world and predict the future. Yet, the language of generalizations (e.g., ``Birds fly'') has received comparatively little attention by formal models, despite its ubiquity in everyday discourse and child-directed speech. Genericity is difficult to formalize because it exhibits extreme flexibility in usage (e.g., not all birds fly) despite looking so simple. I formalize the hypothesis that the core meaning of generalizations in language is simple but underspecified, and that general communicative principles can be used to establish a more precise meaning in context. Using a state-of-the-art probabilistic model of pragmatic reasoning, I examine 3 case studies of generalizations in language: generalizations about events (e.g., John runs), causes (e.g., The block makes the machine play music.), and categories (i.e., generic language, e.g., Birds fly). The model is able to predict graded endorsements from the interaction of diverse prior beliefs about properties with general communicative principles, pointing a way towards more complete models of language and cognition.

\noindent\textbf{Speaking of kinds: How generics convey information about category structure} \\
\noindent\emph{Emily Foster-Hanson, Marjorie Rhodes, Sarah-Jane Leslie}

Generic language (e.g., ``Boys play baseball'', ``Boys like blue'') leads children to assume that particular categories (\textsc{boys}) contain members that are united by deep, intrinsic causal mechanisms \cite{gelman2010effects, rhodes2012cultural}. 
The present studies (N = 120, 5-year-old children) examined why this might be the case. 
We investigated the hypothesis that children expect adults to use generics systematically to refer to categories that are objectively meaningful and reflect natural causal structure \cite<cf.,>{gelman2003preschool, jaswal2007looks}
% that because preschool-age children expect (a) that generics refer to categories \cite{gelman2003preschool}, and (b) that adults usually know the right names for things \cite{jaswal2007looks}, they might assume that adults use generics systematically to refer to categories that are objectively meaningful and reflect natural causal structure. 
A unique feature of this account is that subsequently falsified generics could change how children think about categories.
% even if children later learn that the generics conveyed completely inaccurate content. 
%From this perspective, even if an adult said something wrong about a category, their choice to use a generic could still be consequential.

In these studies, children were exposed to generic sentences about novel categories (e.g., ``Zarpies have striped hair''), which were later contradicted in various ways. 
Some of these contradictions maintained the generic scope of the sentence (e.g., ``Zarpies don't have striped hair''; ``No, that's not right about Zarpies''), whereas others affirmed the property but contradicted the generic scope (e.g., ``No, this Zarpie has striped hair''). 
We found that children held stronger beliefs that members of the category shared intrinsic causal mechanisms upon hearing contradictions that maintained the generic scope than when they heard statements that challenged the generic scope. 
These studies provide evidence that generics influence conceptual development not only through the content they convey, but also because children understand them to refer to meaningful causal clusters in the world.

\noindent\textbf{Is exceptionality something to be taken seriously?  We will try.} 
\noindent\emph{Greg Carlson}

The initial instinct with generics, such as ``Cats chase mice'', is to treat these as expressing a universal truth (in this case, about \textsc{cats}), allowing for some exceptions.
 Exactly how these ``exceptions'' are dealt with formally may vary---e.g., by invoking a notion of ``normality'', %\cite{eckardt1999normal}, 
 circumscription, %\cite{mccarthy1980circumscription}, 
 non-monotonic reasoning, %\cite{asher1995some}, 
 near-universal quantification, %\cite{krifka1995introduction}, 
 attributions to the ``metaphysical'', %\cite{liebesman2011simple}, 
 among others---but the intuition is that, say, a cat that doesn't chase mice (even when presented with the opportunity) does not conform to canonical cat-behavior.  
However, I first argue that generics can be regarded as true even when only minority or even a few of the instances seem to conform to the canonical behavior or property expressed.  
These cases go far beyond the instances previously emphasized (e.g., ``Mosquitos carry the West Nile virus''), and in such instances the notion of non-conforming individuals being regarded as ``exceptions'' seems attenuated if not absent altogether.

Why might exceptionality appear relevant in some generics and not in others? 
%Let us assume, possibly incorrectly, that exceptionality is something to be considered \emph{directly} rather than being regarded as a by-product of some other processes.  
%I examine two potential answers. 
It could be a direct function of \emph{pervasiveness} and a purely graded intuition.
For example, strong intuitions of exceptionality are generated if the ratio of conforming to nonconforming individuals is sufficiently high, and intuitions weaken monotonically as the ratio decreases.
I will present some survey data that bears on this question, as well as observations stemming from recent work by \citeA{leslie2017original}. 
The other option attributes the intuition to a fundamental semantic source: It could be that there are really two or more notions that are conflated under the the label ``generic'' to be untangled. 
Both cross-linguistic work by myself and the work of \citeA{prasada2013conceptual} provide bases for considering this possibility.  
Roughly speaking, the more attributable the regularity to something like inherent structure (a notion to be unpacked), the more likely exceptionality is to arise for the nonconforming instances.

\nocite{shepard1987}

\bibliographystyle{apacite}

\setlength{\bibleftmargin}{.125in}
\setlength{\bibindent}{-\bibleftmargin}
\vspace{-0.1cm}
\bibliography{generalization-symposium}


\end{document}
