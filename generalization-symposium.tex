% 
% Annual Cognitive Science Conference
% Sample LaTeX Two-Page Summary -- Proceedings Format
% 

% Original : Ashwin Ram (ashwin@cc.gatech.edu)       04/01/1994
% Modified : Johanna Moore (jmoore@cs.pitt.edu)      03/17/1995
% Modified : David Noelle (noelle@ucsd.edu)          03/15/1996
% Modified : Pat Langley (langley@cs.stanford.edu)   01/26/1997
% Latex2e corrections by Ramin Charles Nakisa        01/28/1997 
% Modified : Tina Eliassi-Rad (eliassi@cs.wisc.edu)  01/31/1998
% Modified : Trisha Yannuzzi (trisha@ircs.upenn.edu) 12/28/1999 (in process)
% Modified : Mary Ellen Foster (M.E.Foster@ed.ac.uk) 12/11/2000
% Modified : Ken Forbus                              01/23/2004
% Modified : Eli M. Silk (esilk@pitt.edu)            05/24/2005
% Modified : Niels Taatgen (taatgen@cmu.edu)         10/24/2006
% Modified : David Noelle (dnoelle@ucmerced.edu)     11/19/2014

%% Change "letterpaper" in the following line to "a4paper" if you must.

\documentclass[10pt,letterpaper]{article}

\usepackage{cogsci}
\usepackage{pslatex}
\usepackage{apacite}


\title{Generalization, from learning to transmission}
 
\author{{\large \bf Michael Henry Tessler (mtessler@stanford.edu)}, {\large \bf Noah D. Goodman (ngoodman@stanford.edu)}  \\
  Department of Psychology, Stanford University
  \AND {\large \bf Marjorie Rhodes (marjorie.rhodes@nyu.edu)} \\
  Department of Psychology, New York University
   \AND {\large \bf David Danks (ddanks@cmu.edu)} \\
  Department of Philosophy, Carnegie Mellon University
    \AND {\large \bf Gregory Carlson (calrson@ling.rochester.edu)} \\
  Department of Linguistics, University of Rochester
  }


\begin{document}

\maketitle

%\begin{quote}
%\small
%\textbf{Keywords:} 
%add your choice of indexing terms or keywords; kindly use a
%semicolon; between each term
%\end{quote}

\section{Overview}

Learning that an object tends to have a property, an entity tends to exhibit a behavior, or a cause tends to produce an effect is crucial to thrive in our open-ended world. 
Such generalizations can be tricky to acquire.
It thus is crucial that language allow us to communicate these generalizations to each other.
The aim of this symposium is to gather and integrate several distinct empirical and theoretical perspectives on this question, bridging different domains of application.

...

This set of issues cuts across the interests of many different areas of cognitive science: from linguists concerned with the origins of language, to developmental psychologists interested in how children co-create their culture, to computer scientists attempting to build self-sustaining robot collectives. 
Because of this breadth and because recent developments have opened up exciting new sources of data and theory, we are confident this symposium will be of broad interest at Cognitive Science.

The symposium will consist of four talks, described below, by the leaders of the field.
We will close with a panel discussion on some of these issues, led by Noah Goodman.

\noindent\textbf{Title 1} \\
\noindent\emph{Marjorie Rhodes}

First level headings should be in 12~point, initial caps, bold and
centered. Leave one line space above the heading and 1/4~line space
below the heading.

\noindent\textbf{Title 2} \\
\noindent\emph{David Danks}

Second level headings should be 11~point, initial caps, bold, and
flush left. Leave one line space above the heading and 1/4~line
space below the heading.

\noindent\textbf{Title 3} \\
\noindent\emph{Greg Carlson}

Third level headings should be 10~point, initial caps, bold, and flush
left. Leave one line space above the heading, but no space after the
heading.


\noindent\textbf{Communicating generalizations in computational terms} \\
\noindent\emph{Michael Henry Tessler}

Generalizations are the foundation of abstract thought: They allow us to make sense of the world and predict the future. Probabilistic cognitive models formalize generalizations using the Bayesian probability calculus and have been immensely successful at characterizing human thought in computational terms. Yet, the language of generalizations (e.g., Birds fly.) has received comparatively little attention by formal models, despite its ubiquity in everyday discourse and child-directed speech. Genericity is difficult to formalize because it exhibits extreme flexibility in usage (e.g., not all birds fly) despite looking so simple. I formalize the hypothesis that the core meaning of generalizations in language is simple but underspecified, and that general communicative principles can be used to establish a more precise meaning in context. Using a state-of-the-art probabilistic model of pragmatic reasoning, I examine 3 case studies of generalizations in language: generalizations about events (e.g., John runs), causes (e.g., The block makes the machine play music.), and categories (i.e., generic language, e.g., Birds fly). The model is able to predict graded endorsements from the interaction of diverse prior beliefs about properties with general communicative principles, pointing a way towards more complete models of language and cognition.

\bibliographystyle{apacite}

\setlength{\bibleftmargin}{.125in}
\setlength{\bibindent}{-\bibleftmargin}

\bibliography{generalization-symposium}


\end{document}
